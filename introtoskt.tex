%
% INTROTOSKT.SKT
%

\documentclass[12pt]{article}
\usepackage[larger,iitalic,uitalic]{skt}
% Page size compatible both with Letter and A4 papers.
\usepackage[paperheight=279.4mm, paperwidth=210mm]{geometry}
\usepackage{multicol}
\usepackage{xcolor}
\usepackage{mflogo}
\providecommand{\lightgray}{\color{lightgray}}
\usepackage{microtype}
\usepackage{titlesec}

% Tweak section title format to conserve space.
\titleformat{\section}{\large\bfseries}{\thesection}{\parskip}{}
% Make \section*s unnumbered.
\setcounter{secnumdepth}{0}

\DeclareFixedFont{\bigskt}{OT1}{skt}{m}{n}{46mm}

% Dotted hrule.
\makeatletter
\newcommand{\Dotfill}{\leavevmode \cleaders \hb@xt@ .9em{\hss .\hss }\hfill \kern \z@}
\makeatother
\newcommand{\dotrule}{\hspace*{\fill}\parbox{0.3\textwidth}{\Dotfill}\hspace*{\fill}}

\pagenumbering{skt}

{\skt }

%\makeatletter\def\@skt#1{{\bf\small Beta - }{{}\skt\number #1}}\makeatother%%%

%\addtolength{\columnsep}{-44pt}
\parindent 0pt
\addtolength{\parskip}{2mm}
\tolerance=500
\raggedbottom
\hyphenpenalty=9999

% The following commands etc. are used by the sa.myoga table only.
\newcounter{entrycount}
\setcounter{entrycount}{0}
\newcommand{\entrynumber}[1]%
   {\stepcounter{entrycount}%
    \makebox[7mm][r]{\theentrycount}%
    \makebox[1mm][l]{\textsuperscript{#1}}}
\newcommand{\sing}[3]%
   {\mbox{\entrynumber{#1} \makebox[14mm][l]{#2}{#3}}\par}
\newcommand{\dual}[4]%
   {\mbox{\entrynumber{#1} \makebox[14mm][l]{#2}{#3}}\par%
    \hspace*{23.5mm}\mbox{#4}\par}

\title{Sanskrit for \LaTeXe{}\\Version 2\\--\\A Short Introduction and Quick Start Guide}
\author{Charles Wikner (1996-2002)\\Václav Haisman (2016-2025)\\Sumukh Prasad (2025-present)}
\date{}

\begin{document}

\begin{titlepage}
  \maketitle
\end{titlepage}

\tableofcontents
\clearpage

\section{Introduction}%
%
The package contains the font files and pre-processor for printing
Sanskrit text in both {\skti devan\=agar\={\i}} and transliterated Roman
with diacritics.

Some features of the {\skti devan\=agar\={\i}} font:\\[2mm]
\makebox[8mm][c]{$\bullet$}
      Font available in three weights 
      and two slopes.\\ 
\makebox[8mm][c]{$\bullet$}
      The i-hooks connect to the consonants at the correct positions.\\
\makebox[8mm][c]{$\bullet$}
      Accents for all {\skti Veda\/}s ({\skti \d Rgveda}, {\skti S\=amaveda}, 
      {\skti Taittir\={\i}ya}), etc.\\
\makebox[8mm][c]{$\bullet$}
      Accent marking in colour or grey shading.\\
\makebox[8mm][c]{$\bullet$}
      Choice of character forms (e.g.~{\skt A\,\ZN{/}\,`A}) and
      {\skti sa\d myoga} (e.g.~{\skt \ZM{BNz0dcRfA}*.a\,\ZN{/}\,Vca}).\\
\makebox[8mm][c]{$\bullet$}
      Range of intra- and inter-syllable spacing.\\
\makebox[8mm][c]{$\bullet$}
      Dynamic option selection.

Some features of the transliterated Roman:\\[2mm]
\makebox[8mm][c]{$\bullet$}
      Full accent marking and stacking in basic transliteration mode.\\
\makebox[8mm][c]{$\bullet$}
      Four `cases' (as per S.\,M.\,Katre) in technical transliteration mode.\\
\makebox[8mm][c]{$\bullet$}
      Each transliteration mode has four styles: Roman/italic $\times$\ 
      bold/normal. 

This document intends to serve as a short introduction and guide to get started with using \verb|CTAN:pkg/skt| for typesetting documents.

\vfill

\clearpage

\section{Illustrations of Input and {\sktX Devan\=agar\={\i}} Output}%
%
The following brief samples demonstrate the available styles of 
{\skti devan\=agar\={\i}\/} font and their encoding. In this mode upper 
case characters are disallowed. 

Note that the `larger' option has been selected so that, in
this 12pt document, the {\skti devan\=agar\={\i}\/} is printed at 14pt.

\verb+{\+\verb+sktb  te_ja_svi naa_vadhii!tamastu|}+\hspace{2mm}
             {\sktb  .te\ZK{`8}a.j\ZK{`8}a:\ZH{0}{i0//Y7}a.s1va na.\ZK{`8}a:va:Di4a.\ZK{`7}a:ta:ma:s1tua\ZS{12}@A}\\[1mm]
\verb+{\+\verb+skt   te_ja_svi naa_vadhii!tamastu|}+\hspace{2mm}
             {\skt  .te\ZK{`8}a.j\ZK{`8}a:\ZH{0}{i0//Y7}a.s1va na.\ZK{`8}a:va:Di4a.\ZK{`7}a:ta:ma:s1tua\ZS{12}@A}\\[1mm]
\verb+{\+\verb+sktf  te_ja_svi naa_vadhii!tamastu|}+\hspace{2mm}
             {\sktf  .te\ZK{`8}a.j\ZK{`8}a:\ZH{0}{i0//Y7}a.s1va na.\ZK{`8}a:va:Di4a.\ZK{`7}a:ta:ma:s1tua\ZS{12}@A}\\[1mm]
\verb+{\+\verb+sktbs te_ja_svi naa_vadhii!tamastu|}+\hspace{2mm}
             {\sktbs .te\ZK{`8}a.j\ZK{`8}a:\ZH{0}{i0//Y7}a.s1va na.\ZK{`8}a:va:Di4a.\ZK{`7}a:ta:ma:s1tua\ZS{12}@A}\\[1mm]
\verb+{\+\verb+skts  te_ja_svi naa_vadhii!tamastu|}+\hspace{2mm}
             {\skts  .te\ZK{`8}a.j\ZK{`8}a:\ZH{0}{i0//Y7}a.s1va na.\ZK{`8}a:va:Di4a.\ZK{`7}a:ta:ma:s1tua\ZS{12}@A}\\[1mm]
\verb+{\+\verb+sktfs te_ja_svi naa_vadhii!tamastu|}+\hspace{2mm}
             {\sktfs .te\ZK{`8}a.j\ZK{`8}a:\ZH{0}{i0//Y7}a.s1va na.\ZK{`8}a:va:Di4a.\ZK{`7}a:ta:ma:s1tua\ZS{12}@A}


The intra- and inter-character spacing options give six steps of horizontal
density; the extremes are:
\vspace{-1mm}
\begin{center}
{\skt  .nEa;va ;Y2a;k\ZH{-12}{M}+:Y4a;.ca;tk+.=+ea;mi6a;a;Y3a;ta yua;k2+:ea ma;nyea;ta ta:t1va;Y2a;va;t,a\ZS{12}@A}\\[2mm]
{\skt  .nEa.va ;i9ak\ZH{-12}{M}+Y1a.ca.tk;=.ea.mi3aa.Y0ata yua.k2;ea ma.nyea.ta tat1va.i9ava.t,a\ZS{12}@A}
\end{center}


\section{Illustrations of Basic Transliteration}%
%
The same input may also be used to produce transliterated output.
Note that the `\verb+iitalic+' option is selected, but not `\verb+xitalic+'.

{\skt }
\verb+{\+\verb+sktx te_ja_svi naa_vadhii!tamastu|}+\hspace{.8mm}
             {\sktx t\ZA{2}{e}j\ZA{2}{a}svi n\ZA{2}{\=a}vadh\ZA{1}{\={\i}}tamastu{\upshape\,$\mid$}}

\verb+{\+\verb+sktX te_ja_svi naa_vadhii!tamastu|}+\hspace{.8mm}
             {\sktX t\ZA{2}{e}j\ZA{2}{a}svi n\ZA{2}{\=a}vadh\ZA{1}{\={\i}}tamastu{\upshape\boldmath\,$\mid$}}

\verb+{\+\verb+skti te_ja_svi naa_vadhii!tamastu|}+\hspace{.8mm}
             {\skti t\ZA{2}{e}j\ZA{2}{a}svi n\ZA{2}{\=a}vadh\ZA{1}{\={\i}}tamastu{\upshape\,$\mid$}}

\verb+{\+\verb+sktI te_ja_svi naa_vadhii!tamastu|}+\hspace{.8mm}
             {\sktI t\ZA{2}{e}j\ZA{2}{a}svi n\ZA{2}{\=a}vadh\ZA{1}{\={\i}}tamastu{\upshape\boldmath\,$\mid$}}

It can be seen from the above examples that \verb+sktX+ is a boldface
version of \verb+sktx+ (i.e.~also upright), and that \verb+sktI+ is
similarly a boldface version of \verb+skti+ (also italic).
The only difference between \verb+sktx/X+ and \verb+skti/I+ is that
they may be independently selected as italic in the style file options.

\pagebreak

The Western accent marking system may also be used:\\[2mm]
\verb+{\+\verb+skti tejasvi' naava'dhii`tamastu|}+\hspace{.8mm}
             {\skti tejasv\ZA{8}{{\i}} n\=av\ZA{8}{a}dh\ZA{7}{\={\i}}tamastu{\upshape\,$\mid$}}

In this basic transliteration mode (\verb+sktx/X+ or \verb+skti/I+) uppercase 
characters are allowed: these do not affect the transliteration encoding, 
but facilitate the printing of capital letters, for example:\\[2mm]
\verb+{\+\verb+sktI  Bhaarate Raamo vasati}+\hspace{5mm}
                   {\sktI  Bh\=arate R\=amo vasati}
\vspace{4mm}

\section{Illustrations of Technical Transliteration}%
%
As in the basic transliteration mode, there are four font styles selected
through \verb+sktt/T+ or \verb+sktu/U+; however, in this mode only the
Western accent marking is allowed. This technical (grammatical) mode
follows the style of S\,M\,Katre in his translation of the 
{\skti A\d s\d t\=adhy\=ay\={\i}}, having four `cases' of letters.

Uppercase is used to indicate {\sktT \ZY{I}\ZY{T}\/} letters 
(e.g.~{\skti praty\=ah\=ara\/} {\sktT a\ZX{C}\/}): note that in this mode
a {\skti mah\=apr\=a\d na spar\'sa\/} (e.g.~{\skti Bha\/}) must have the 
`{\skti h\/}' in the same case as the previous letter. In the example 
given above, \verb+Bhaarate+ would produce an error: it needs to be 
\verb+bhaarate+ or \verb+BHaarate+. The uppercase letters in this mode 
are slightly smaller than those in the basic transliteration mode.

In this mode only, letters may also be preceded by the underscore character
`\verb+_+'. The effect that this has depends upon the case of the following
letter: if the letter is lowercase, it will be underlined to indicate that
its presence is for the sake of pronunciation only (e.g.~{\sktT jh\ZW{a}\ZX{L}}\/);
before an uppercase letter, it will reduce the size of the uppercase letter
to the height of the lowercase letters to indicate technical words
(e.g.~{\sktT \ZY{I}\ZY{T}\/}).
For example, the {\skti P\=a\d nini s\=utra\/}\\[2mm]
{\skt ..ca:ja;eaH\ZS{4} k\ZH{-12}{u} ;Y3a;Ga;NNya;ta;eaH\ZS{4}\ZS{12}@A\ZS{6}@A 7\ZS{12}@A 3\ZS{12}@A 52\ZS{12}@A\ZS{6}@A}\\[2mm]
may be encoded in the technical mode as\\[3mm]
\verb+{\+\verb+sktT c_a-j-o.h kU GH_I_T=.NyaT-o.h}+ to produce\\[2mm]
{\sktT c\ZW{a}\ZN{-}j\ZN{-}o\d h k\ZX{U} \ZX{GH}\ZY{I}\ZY{T}\ZN{=}\ZX{\d N}ya\ZX{T}\ZN{-}o\d h}\\[3mm]
This translates as: A substitute {\sktT k\ZX{U}\/} replaces {\sktT c\/} or 
{\sktT j\/} before {\sktT \ZX{\d N}ya\ZX{T}\/} or [affixes]\newline\hspace*{35mm}with 
{\sktT \ZX{GH}\/} as an {\sktT \ZY{I}\ZY{T}\/} marker.

\vfill

\clearpage

%%%%%%%%%%%%%%%%%%%%%%%%%%%%%%%%%%%%%%%%%%%%%%%%%%%%%%%%%%%%%%%%%%%%%%%%%

\section{Getting Started with Typesetting Documents with Sanskrit for \LaTeXe{} Version 2}

The pre-processor contained in \verb|skt.c|processes the source text file (with a default \verb+.skt+
filename extension) to produce a file suitable for \LaTeXe\ (with a default
\verb+.tex+ extension). For example, the command \verb+skt test+ will convert
\verb+test.skt+ to \verb+test.tex+; fuller filenames may be used, 
e.g.~\verb+skt foo.bar+ will produce \verb+foo.tex+, and 
\verb+skt foo.bar far.boo+ will produce \verb+far.boo+ from \verb+foo.bar+.
If no filename is specified, you will be prompted for input and output
filenames. The pre-processor uses the Velthuis system of transliteration to process and convert text.

The program passes the text from the input file to the output file
unaltered, until it finds the string `\verb+{+\verb+\skt+'. 
It then checks the following character(s) for the acceptable modifiers
\verb+b f s bs fs x X i I t T u+ or \verb+U+, and this must be followed by
a space character: if this test fails it reports an error, otherwise
the program converts the following input text to {\skti devan\=agar\={\i}\/}
or transliteration format as required, until the matching `\verb+}+'.
The cycle is then repeated.

Within the \verb+skt+ text, the following punctuation characters will be
passed to the output as roman:\label{ref:punct}

\begin{center}
\framebox{\texttt{ ( ) * + , - / : ; = ? }}
\end{center}

Furthermore, two successive periods `..' will be passed to the output as 
a single roman period.  This technique of repeating the character twice 
to produce a single roman character applies to the four characters \verb|[`']| 
as well; and the two-character string `.!' will pass through as a single 
roman exclamation mark.  You will need to use \verb+{+\verb+\sktx ``a'{}''}+
to produce {\sktx `\ZA{8}{a}'} (see Supplementary Notes for explanation).

\LaTeX\ command strings embedded within the \verb+skt+ text will be
passed to the output file unchanged\,---\,but beware: command parameters
will be converted\,! Thus commands without parameters (e.g.~\verb+\clearpage+)
are safe, but commands with parameters (e.g.~\verb+\hspace{3mm}+) will
cause problems unless the parameters are meant to be converted
(e.g.~\verb+\underline{naaman}+).

\begin{center}
\framebox{\begin{minipage}{100mm}
\textbf{Note: }This program will stop passing text to the output file
as soon as an error in the source file is detected; it will however,
continue processing the input file until ten or more errors are 
encountered.
\end{minipage}}
\end{center}

For example, an end-to-end workflow can be:

\begin{enumerate}
	\item Download the repository, compile \verb|skt.c|, and move the compiled program to the same location as your \verb|*.skt| documents.
	
	\begin{itemize}
		\item Your \verb|*.skt| documents MUST have \verb|\usepackage[options]{skt}| in the preamble.
	\end{itemize}
	
	\item Use \verb|./skt foo.skt| to produce a corresponding \LaTeXe file.
	
	\item Use \verb|pdflatex foo.tex| to produce a PDF.

	\item Edit your \verb|*.skt| file and repeeat steps (2) and (3) until the desired result is achieved.
\end{enumerate}

\vfill

\clearpage

%%%%%%%%%%%%%%%%%%%%%%%%%%%%%%%%%%%%%%%%%%%%%%%%%%%%%%%%%%%%%%%%%%%%%%%%%

\section{Examples in Typesetting {\sktI devan\=agar\={\i}}}


\begin{minipage}[t]{\linewidth}
\small
\verb+{+\verb+\+\texttt{sktf}\\
\texttt{"suddhabrahmaparaatpara raam ||1||}\\
\texttt{kaalaatmakaparame"svara raam ||2||}\\
\texttt{"se.satalpasukhanidrita raam ||3||}\\
\texttt{brahmaadyaamarapraarthita raam ||4||}\\
\verb|}|
\end{minipage}


\begin{minipage}[t]{\linewidth}
\small
{\sktf 
Zua:;d\ZM{t0Dj0b};b.ra;h2;pa:=+a;tpa:= .=+a;m,a \ZS{12}@A\ZS{6}@A1\ZS{12}@A\ZS{6}@A \\
k+:a;l+.a;tma;k+:pa:=+mea:(\ZM{dGu};a:= .=+a;m,a \ZS{12}@A\ZS{6}@A2\ZS{12}@A\ZS{6}@A \\
Zea;Sa;ta;l1pa;sua;Ka;Y4a;na;Y2a;d\ZP{-8}{-4}{@R}+ta .=+a;m,a \ZS{12}@A\ZS{6}@A3\ZS{12}@A\ZS{6}@A \\
b.ra;h2;a;d\ZM{fByF0E};a;a;ma:=+pra;a;Y4a;TRa;ta .=+a;m,a \ZS{12}@A\ZS{6}@A4\ZS{12}@A\ZS{6}@A \\
}
\end{minipage}
\\
\\

\begin{minipage}[t]{\linewidth}
\small
\verb+{+\verb+\+\texttt{sktfs}\\
\texttt{ca.n.dakira.nakulama.n.dana raam ||5||} \\
\texttt{"sriimadda"sarathanandana raam ||6||} \\
\texttt{kausalyaasukhavardhana raam ||7||} \\
\texttt{vi"svaamitrapriyadhana raam ||8||} \\
\verb|}|
\end{minipage}


\begin{minipage}[t]{\linewidth}
\small
{\sktfs 
..ca;Nq+.Y2a;k+.=+Na;k\ZH{-12}{u}+:l+.ma;Nq+.na .=+a;m,a \ZS{12}@A\ZS{6}@A5\ZS{12}@A\ZS{6}@A \\
(ri6a;a;ma;d1+Za:=+Ta;na;nd;na .=+a;m,a \ZS{12}@A\ZS{6}@A6\ZS{12}@A\ZS{6}@A \\
k+:Ea;sa;l1ya;a;sua;Ka;va;DRa;na .=+a;m,a \ZS{12}@A\ZS{6}@A7\ZS{12}@A\ZS{6}@A \\
;Y2a;va:(\ZM{dGu};a;a;Y6a;ma:t3a;Y2a;pra;ya;Da;na .=+a;m,a \ZS{12}@A\ZS{6}@A8\ZS{12}@A\ZS{6}@A \\
}
\end{minipage}
\\
\\


\begin{minipage}[t]{\linewidth}
\small
\verb+{+\verb+\+\texttt{skt}\\
\texttt{ghorataa.takaaghaataka raam ||9||} \\
\texttt{maariicaadinipaataka raam ||10||} \\
\texttt{kau"sikamakhasa.mrak.saka raam ||11||} \\
\texttt{"sriimadahalyoddhaaraka raam ||12||} \\
\verb|}|
\end{minipage}


\begin{minipage}[t]{\linewidth}
\small
{\skt 
;Ga;ea:=+ta;a;f;k+:a;Ga;a;ta;k .=+a;m,a \ZS{12}@A\ZS{6}@A9\ZS{12}@A\ZS{6}@A \\
ma;a:=\ZH{-6}{i7}+a;.ca;a;Y2a;d;Y4a;na;pa;a;ta;k .=+a;m,a \ZS{12}@A\ZS{6}@A10\ZS{12}@A\ZS{6}@A \\
k+:Ea;Y5a;Za;k+:ma;Ka;sMa:=+[a;k .=+a;m,a \ZS{12}@A\ZS{6}@A11\ZS{12}@A\ZS{6}@A \\
(ri6a;a;ma;d;h;l1ya;ea:;d\ZM{t0Dj0b};a:=+k .=+a;m,a \ZS{12}@A\ZS{6}@A12\ZS{12}@A\ZS{6}@A \\
}
\end{minipage}
\\
\\

\begin{minipage}[t]{\linewidth}
\small
\verb+{+\verb+\+\texttt{skts}\\
\texttt{gautamamunisa.mpuujita raam ||13||} \\
\texttt{suramunivaraga.nasa.mstuta raam ||14||} \\
\texttt{naavikadhaavitam.rdupada raam ||15||} \\
\texttt{mithilaapurajanamohaka raam ||16||} \\
\verb|}|
\end{minipage}


\begin{minipage}[t]{\linewidth}
\small
{\skts 
ga;Ea;ta;ma;mua;Y4a;na;sMa;pUa;Y6a:ja;ta .=+a;m,a \ZS{12}@A\ZS{6}@A13\ZS{12}@A\ZS{6}@A \\
.sua:=+mua;Y4a;na;va:=+ga;Na;sMa;s1tua;ta .=+a;m,a \ZS{12}@A\ZS{6}@A14\ZS{12}@A\ZS{6}@A \\
na;a;Y2a;va;k+:Da;a;Y2a;va;ta;m\ZV{2}{x}a;d\ZH{-10}{u};pa;d .=+a;m,a \ZS{12}@A\ZS{6}@A15\ZS{12}@A\ZS{6}@A \\
;Y6a;ma;Y4a;Ta;l+.a;pua:=;ja;na;ma;ea;h;k .=+a;m,a \ZS{12}@A\ZS{6}@A16\ZS{12}@A\ZS{6}@A \\
}
\end{minipage}
\\
\\


\begin{minipage}[t]{\linewidth}
\small
\verb+{+\verb+\+\texttt{sktb}\\
\texttt{videhamaanasara~njaka raam ||17||} \\
\texttt{trya.mbakakaarmukabha~njaka raam ||18||} \\
\texttt{siitaarpitavaramaalika raam ||19||} \\
\texttt{k.rtavaivaahikakautuka raam ||20||} \\
\verb|}|
\end{minipage}


\begin{minipage}[t]{\linewidth}
\small
{\sktb 
;Y2a;va;d\ZH{-6}{e};h;ma;a;na;sa:=;\ZM{BNz0djR0ARdA}*+;k .=+a;m,a \ZS{12}@A\ZS{6}@A17\ZS{12}@A\ZS{6}@A \\
t3yMa;ba;k+:k+:a;mRua;k+:Ba:\ZM{BNz0djR0ARdA}*+;k .=+a;m,a \ZS{12}@A\ZS{6}@A18\ZS{12}@A\ZS{6}@A \\
.si6a;a;ta;a;Y2a;pRa;ta;va:=+ma;a;Y4a;l+.k .=+a;m,a \ZS{12}@A\ZS{6}@A19\ZS{12}@A\ZS{6}@A \\
k\ZH{-12}{\ZV{4}{x}}+:ta;vEa;va;a;Y2a;h;k+:k+:Ea;tua;k .=+a;m,a \ZS{12}@A\ZS{6}@A20\ZS{12}@A\ZS{6}@A \\
}
\end{minipage}
\\
\\

\begin{minipage}[t]{\linewidth}
\small
\verb+{+\verb+\+\texttt{sktbs}\\
\texttt{bhaargavadarpavinaa"saka raam ||21||} \\
\texttt{"sriimadayodhyaapaalaka raam ||22||} \\
\verb|}|
\end{minipage}


\begin{minipage}[t]{\linewidth}
\small
{\sktbs 
Ba;a;gRa;va;d;pRa;Y2a;va;na;a;Za;k .=+a;m,a \ZS{12}@A\ZS{6}@A21\ZS{12}@A\ZS{6}@A \\
(ri6a;a;ma;d;ya;ea;Dya;a;pa;a;l+.k .=+a;m,a \ZS{12}@A\ZS{6}@A22\ZS{12}@A\ZS{6}@A \\
}
\end{minipage}


\section{Examples in Typesetting Transliterated {\sktI devan\=agar\={\i}}}

\begin{minipage}[t]{\linewidth}
\small
\verb+{+\verb+\+\texttt{sktx}\\
\texttt{mudaakaraattamodaka.m sadaa vimuktisaadhaka.m} \\
\texttt{kalaadharaavata.msaka.m vilaasilokarak.sakam |} \\
\texttt{anaayakaikanaayaka.m vinaa"sitebhadaityaka.m} \\
\texttt{nataa"subhaa"sunaa"saka.m namaami ta.m vinaayakam ||1||} \\
\verb|}|
\end{minipage}


\begin{minipage}[t]{\linewidth}
\small
{\sktx 
mud\=akar\=attamodaka\d m sad\=a vimuktis\=adhaka\d m \\
kal\=adhar\=avata\d msaka\d m vil\=asilokarak\d sakam {\upshape\,$\mid$} \\
an\=ayakaikan\=ayaka\d m vin\=a\'sitebhadaityaka\d m \\
nat\=a\'subh\=a\'sun\=a\'saka\d m nam\=ami ta\d m vin\=ayakam {\upshape\,$\mid\mid$}1{\upshape\,$\mid\mid$} \\
}
\end{minipage}
\\
\\

\begin{minipage}[t]{\linewidth}
\small
\verb+{+\verb+\+\texttt{sktX}\\
\texttt{natetaraatibhiikara.m navoditaarkabhaasvara.m} \\
\texttt{namatsuraarinirjara.m nataadhikaapaduddharam |} \\
\texttt{sure"svara.m nidhii"svara.m gaje"svara.m ga.ne"svara.m} \\
\texttt{mahe"svara.m tamaa"sraye paraatpara.m nirantaram ||2||} \\
\verb|}|
\end{minipage}


\begin{minipage}[t]{\linewidth}
\small
{\sktX 
natetar\=atibh\={\i}kara\d m navodit\=arkabh\=asvara\d m \\
namatsur\=arinirjara\d m nat\=adhik\=apaduddharam {\upshape\boldmath\,$\mid$} \\
sure\'svara\d m nidh\={\i}\'svara\d m gaje\'svara\d m ga\d ne\'svara\d m \\
mahe\'svara\d m tam\=a\'sraye par\=atpara\d m nirantaram {\upshape\boldmath\,$\mid\mid$}2{\upshape\boldmath\,$\mid\mid$} \\
}
\end{minipage}
\\
\\


\begin{minipage}[t]{\linewidth}
\small
\verb+{+\verb+\+\texttt{skti}\\
\texttt{samastaloka"sa.mkara.m nirastadaityaku~njara.m} \\
\texttt{daretarodara.m vara.m varebhavaktramak.saram |} \\
\texttt{k.rpaakara.m k.samaakara.m mudaakara.m ya"saskara.m} \\
\texttt{manaskara.m namask.rtaa.m namaskaromi bhaasvaram ||3||} \\
\verb|}|
\end{minipage}


\begin{minipage}[t]{\linewidth}
\small
{\skti 
samastaloka\'sa\d mkara\d m nirastadaityaku\~njara\d m \\
daretarodara\d m vara\d m varebhavaktramak\d saram {\upshape\,$\mid$} \\
kr\llap{\d{\kern.51em}}p\=akara\d m k\d sam\=akara\d m mud\=akara\d m ya\'saskara\d m \\
manaskara\d m namaskr\llap{\d{\kern.51em}}t\=a\d m namaskaromi bh\=asvaram {\upshape\,$\mid\mid$}3{\upshape\,$\mid\mid$} \\
}
\end{minipage}
\\
\\

\begin{minipage}[t]{\linewidth}
\small
\verb+{+\verb+\+\texttt{sktI}\\
\texttt{aki.mcanaartimaarjana.m cirantanoktibhaajana.m} \\
\texttt{puraaripuurvanandana.m suraarigarvacarva.nam |} \\
\texttt{prapa~ncanaa"sabhii.sa.na.m dhana.mjayaadibhuu.sa.nam} \\
\texttt{kapoladaanavaara.na.m bhaje puraa.navaara.nam ||4||} \\
\verb|}|
\end{minipage}


\begin{minipage}[t]{\linewidth}
\small
{\sktI 
aki\d mcan\=artim\=arjana\d m cirantanoktibh\=ajana\d m \\
pur\=arip\=urvanandana\d m sur\=arigarvacarva\d nam {\upshape\boldmath\,$\mid$} \\
prapa\~ncan\=a\'sabh\={\i}\d sa\d na\d m dhana\d mjay\=adibh\=u\d sa\d nam \\
kapolad\=anav\=ara\d na\d m bhaje pur\=a\d nav\=ara\d nam {\upshape\boldmath\,$\mid\mid$}4{\upshape\boldmath\,$\mid\mid$} \\
}
\end{minipage}


\vfill



\clearpage

\hspace*{10mm}%
\begin{minipage}{122mm}
\addtolength{\parskip}{2mm}
\renewcommand{\baselinestretch}{1.1}

\section{Sample Text from {\sktX \d Rgveda 10.125.}}
\vspace{10mm}

\label{rgtext}
{\skt 
`A\ZH{-6}{\ZK{`8}};h\ZH{-6}{M} 
 .r8\ZH{-4}{\ZK{`8}}+:d\ZP{-8}{-4}{@R}\ZH{-6}{e}+Y6a;B\ZK{`8}a;vRa;s\ZK{`7}ua;Y6a;Ba;(\ZM{rac0FI}a:=+a;m1y\ZK{`8}a;h;ma;\ZK{`7}a;Y2a;d\ZH{-6}{\ZK{`8}};tyEa:r8\ZH{-4}{\ZK{`8}}+:ta 
 ;Y2a;v\ZK{`8}a:(\ZM{dGu};a;d\ZH{-6}{e}\ZH{-6}{\ZK{`7}};vEaH\ZS{4}\ZS{12}@A\\
`A\ZH{-6}{\ZK{`8}};h\ZH{-6}{M} ;Y6a;m\ZK{`8}a:t3a;a;va:r8\ZH{-10}{\ZK{`7}}+:`Na;e\ZK{`8}a;Ba;a
 ;Y2a;b\ZK{`7}a;Ba;m1yR\ZK{`8}a;h;Y6a;m\ZK{`7}a;nd\ZP{-8}{-4}{@R}+\ZK{`8}a;\ZM{0NgFRI0FnIFe}+;a\ZH{-6}{i6};a 
 `A\ZH{-6}{\ZK{`8}};h;m\ZK{`8}a;\ZH{0}{i0Y7}a:(\ZM{dGu};a;na;e\ZK{`8}a;Ba;a\ZS{12}@A\ZS{6}@A 1\ZS{12}@A\ZS{6}@A

`A\ZH{-6}{\ZK{`8}};h\ZH{-6}{M} .sa;ea;m\ZK{`7}a;ma;a;h\ZP{-4}{-4}{\ZK{`8}};na;sM\ZV{8}{\ZK{`7}}a ;Y2a;ba;Ba;m1yR\ZK{`8}a;h\ZH{-6}{M} 
 tva;S1;\ZK{`7}a:=+mu\ZV{-8}{\ZK{`8}}a;ta :pU\ZV{-8}{\ZK{`8}}a;Sa;`NM\ZK{`8}a Ba;g\ZK{`7}a;m,a\ZS{12}@A\\
`A\ZH{-6}{\ZK{`8}};h\ZH{-6}{M} d\ZH{-6}{\ZK{`7}};Da;a;Y6a;m\ZK{`8}a d\ZP{-8}{-4}{@R}+Y2a;v\ZK{`7}a;`NMa h\ZP{-4}{-4}{\ZK{`8}};Y2a;va;Sm\ZK{`7}a;tea 
 .sua;pra;\ZK{`8}a;v.ye\ZK{`8}a\ZK{3\ZH{-12}{`7}\ZH{-8}{`8}} ya:j\ZK{`7}a;ma;a;na;a;ya .sua;nv\ZK{`8}a;tea\ZS{12}@A\ZS{6}@A 2\ZS{12}@A\ZS{6}@A

`A\ZH{-6}{\ZK{`8}};h\ZH{-6}{M} .=+a;S1\ZH{-10}{r1}\ZH{-6}{i6};\ZK{`7}a .sM\ZK{`8}a;ga;m\ZK{`7}a;ni6a;\ZK{`8}a va;s\ZK{`7}Ua;na;Ma 
 ;Y4a;.ca;Y2a;k\ZH{-8}{\ZK{`8}}+:tua;Si6a;\ZK{`7}a :pra;T\ZK{`8}a;ma;a y\ZK{`8}a::]i6a;a;ya;\ZK{`7}a;na;a;m,a\ZS{12}@A\\
ta;Ma ma;\ZK{`7}a :d\ZH{-6}{e}\ZH{-6}{\ZK{`8}};va;a v.y\ZK{`7}a;d;DuaH\ZS{4} :pua:r8\ZH{-4}{\ZK{`8}}+.t3a;a 
 BUa;i8a:=\ZH{-6}{\ZK{`7}}+s1Ta;a:t3a;M\ZK{`8}a BUa;ya;R\ZV{8}{\ZK{`7}}a;ve\ZK{`8}a;Za;y\ZK{`7}a;nti6a;a;m,a\ZS{12}@A\ZS{6}@A 3\ZS{12}@A\ZS{6}@A

ma;ya;\ZK{`8}a .sa;ea `A;n1\ZK{`7}a;ma;Y5a:t1\ZK{`8}a ya;ea ;Y2a;v\ZK{`8}a;pa;Zy\ZK{`7}a;Y3a;t\ZK{`8}a yaH\ZS{4} 
 :pra;a;Y5a;`N\ZK{`7}a;Y3a;t\ZK{`8}a ya I\ZH{-6}{R2}\ZV{8}{\ZH{-6}{\ZK{`7}}} Zx\ZV{-12}{\ZK{`8}}a;`Na;ea;tyu\ZV{-8}{\ZK{`8}}a;k2+:m,a\ZS{12}@A\\
`A\ZH{-6}{\ZK{`8}};m\ZK{`8}a;nta;va;e\ZK{`8}a ma;Ma ta o+.p\ZK{`7}a ;Y6a;[a;ya;\ZH{0}{i0////Y7}a;nta (ru\ZV{-8}{\ZK{`8}}a;Y3a;Da 
 (r\ZK{`7}ua;ta (ra;Y4a:;d\ZM{t0Dj0b}\ZP{-6}{-2}{\ZK{`8}};vMa .te\ZK{`7}a va;d;a;Y6a;ma\ZS{12}@A\ZS{6}@A 4\ZS{12}@A\ZS{6}@A}

\renewcommand{\ZK}[1]{{\lightgray #1}}

{\skt 
`A\ZH{-6}{\ZK{`8}};h;me\ZK{`8}a;va .s1v\ZK{`8}a;ya;Y6a;m\ZK{`8}a;d\ZH{-6}{M} v\ZK{`7}a;d;a;Y6a;m\ZK{`8}a .jua;S1\ZH{-6}{M}\ZV{8}{\ZH{-6}{\ZK{`7}}} 
 :d\ZH{-6}{e}\ZH{-6}{\ZK{`8}};vea;Y6a;B\ZK{`7}a:r8\ZH{-4}{\ZK{`8}}+:ta ma;a;n\ZK{`7}ua;Sea;Y6a;BaH\ZS{4}\ZS{12}@A\\
yMa k+:\ZK{`8}a;ma;ye\ZK{`8}a tMa;t\ZK{`7}a;mu\ZV{-8}{\ZK{`8}}a;gr4Ma k\ZH{-12}{\ZK{`7}}\ZH{-12}{\ZV{4}{x}}+:`Na;ea;Y6a;ma tMa
 b.r\ZK{`8}a;h2;a;`Na;M\ZK{`8}a ta;a;m\ZV{2}{x}a;Y2a;SM\ZK{`8}a tMa .s\ZK{`7}ua;me\ZK{`8}a;Da;a;m,a\ZS{12}@A\ZS{6}@A 5\ZS{12}@A\ZS{6}@A

`A\ZH{-6}{\ZK{`8}};h\ZH{-6}{M} .r8\ZH{-4}{\ZK{`8}}+:d\ZP{-8}{-4}{@R}+a;y\ZK{`8}a ;Da;nu\ZV{-8}{\ZK{`8}}a:=+a t\ZK{`7}a;na;ea;Y6a;ma 
 b.ra;h2\ZP{-6}{-4}{\ZK{`8}};Y2a;d\ZM{oau};Se\ZK{`8}a Za:=\ZH{-6}{\ZK{`7}}+ve\ZK{`8}a h;nt\ZK{`8}a;va;a o\ZH{-6}{\ZK{`7}}\ZS{12}@A\\
`A\ZH{-6}{\ZK{`8}};h\ZH{-6}{M} .ja;na;\ZK{`7}a;ya .s\ZK{`8}a;m\ZK{`7}a;d\ZH{-6}{M} k\ZH{-12}{\ZV{4}{x}}+:`Na;ea;m1y\ZK{`8}a;h\ZH{-6}{M} 
 d\ZM{fByF0E};a;a;va;\ZK{`7}a;p\ZV{4}{x}a;Y4a;T\ZK{`8}a;vi6a;a `A;a ;Y2a;v\ZK{`7}a;vea;Za\ZS{12}@A\ZS{6}@A 6\ZS{12}@A\ZS{6}@A

`A\ZH{-6}{\ZK{`8}};h\ZH{-6}{M} .s\ZK{`7}ua;vea ;Y2a;p\ZK{`8}a;ta:=\ZH{-6}{\ZK{`7}}+ma;s1ya mU\ZV{-8}{\ZK{`8}}a;DRa;nm1\ZM{cLe}.a;m\ZK{`8}a 
 ya;ea;Y4a;n\ZK{`7}a:=\ZH{-2}{\ZK{`8}}+ps1va\ZK{1\ZH{-12}{`7}\ZH{-10}{`8}}+ntaH\ZS{4} .s\ZK{`7}a;mu\ZV{-8}{\ZK{`8}}a;d\ZP{-8}{-4}{@R}\ZH{-6}{e}\ZS{12}@A\\
ta;ta;e\ZK{`8}a ;Y2a;va ;Y3a;t\ZK{`7}a;S2\ZH{-6}{e}\ZH{-6}{\ZK{`8}} Bua;v\ZK{`8}a;na;a;nu\ZV{-8}{\ZK{`8}}a ;Y2a;va:(\ZM{dGu};a;e\ZK{`8}a;ta;a;mMUa 
 d\ZM{fByF0E};a;Ma v\ZK{`8}a;SmRa;`Na;ea;p\ZK{`7}a .s1\ZM{cLe}.p\ZV{4}{x}a;Za;a;Y6a;ma\ZS{12}@A\ZS{6}@A 7\ZS{12}@A\ZS{6}@A

`A\ZH{-6}{\ZK{`8}};h;me\ZK{`8}a;va va;a;t\ZK{`7}a I+.v\ZK{`8}a :pra 
 va;\ZK{`7}a;m1ya;\ZK{`8}a:=+B\ZK{`7}a;ma;a;`Na;\ZK{`8}a 
 Bua;v\ZK{`7}a;na;a;Y4a;n\ZK{`8}a ;Y2a;va:(\ZM{dGu};\ZK{`7}a\ZS{12}@A\\
:p\ZK{`8}a:=+ea ;Y2a;d\ZH{-6}{\ZK{`8}};va;a :p\ZK{`8}a:= O;\ZH{-6}{\ZK{`8}};na;a :p\ZK{`7}\ZV{4}{x}a;Y4a;Ta;v.yEa;ta;a;v\ZK{`7}a;ti6a;a 
 ma;Y2a;h\ZP{-4}{-4}{\ZK{`8}};na;a .sMa b\ZK{`7}a;BUa;va\ZS{12}@A\ZS{6}@A 8\ZS{12}@A\ZS{6}@A

}

\renewcommand{\baselinestretch}{1}
\addtolength{\parskip}{-2mm}
\end{minipage}

\clearpage

\clearpage

\hspace*{10mm}%
\begin{minipage}{122mm}
\addtolength{\parskip}{4mm}
\renewcommand{\baselinestretch}{1.2}

\section{Sample Text from {\sktX S\=amaveda}}
\vspace{6mm}

{\sktI S\=amaveda Sa\d mhit\=a} ({\skti Kauthuma\/}), verses 523--528.
\vspace{15mm}

{\skt }

{\skt 
:pr\ZV{12}{\ZK{`1}}a tua d\ZP{-8}{-4}{@R}\ZV{12}{\ZH{-8}{\ZK{`4}}};v\ZV{12}{\ZK{`3}}a :p\ZV{12}{\ZK{`2}}a:i7a.=\ZV{12}{\ZH{-6}{\ZK{`3}}} k+.e\ZV{12}{\ZK{`2}}a:ZM\ZV{12}{\ZK{`3}}a ;Y3a:n\ZV{12}{\ZK{`1}}a 
 :Si5a:\ZV{12}{\ZK{`2}}a:d\ZV{12}{\ZH{-6}{\ZK{`3}}} n\ZV{12}{\ZK{`1}}\ZV{4}{x}a:Y5a:B\ZV{12}{\ZK{`2}}aH\ZS{2} :pua:na:\ZV{12}{\ZK{`3}}a:na:e\ZV{12}{\ZK{`2}}a `A\ZV{12}{\ZH{-6}{\ZK{`3}}}:Y5a:B\ZV{12}{\ZK{`1}}a 
 va:a.j\ZV{12}{\ZH{-4}{\ZK{`4}}}a:ma:SRa\ZS{12}@A \\
`A\ZV{12}{\ZH{-6}{\ZK{`2}}}:ZvM\ZV{12}{\ZK{`3}}a n\ZV{12}{\ZK{`1}}a tva:\ZV{12}{\ZK{`2}}a va:\ZV{12}{\ZK{`3}}a:Y5a.j\ZV{12}{\ZK{`1}}a:nM\ZV{12}{\ZK{`2}}a 
 m\ZV{12}{\ZK{`3}}a.jR\ZV{12}{\ZK{`2}}a:y\ZV{12}{\ZK{`3}}a:nta:e\ZV{12}{\ZK{`1}}aY:.cC+\ZV{12}{\ZK{`2}}a v\ZV{12}{\ZK{`3}}a:h\ZH{-6}{i5}:R\ZV{12}{\ZK{`1}}a 
 .=\ZV{12}{\ZH{-6}{\ZK{`2}}};Z\ZV{12}{\ZK{`3}}a:na:\ZV{12}{\ZK{`1}}a:Y5a:B\ZV{12}{\ZK{`2}}a:nRa:ya:\ZH{0}{i0///Y7}a:nta\ZS{12}@A\ZS{6}@A 523\ZS{12}@A\ZS{6}@A

:pr\ZV{12}{\ZK{`1}}a k+.a:v.y\ZV{12}{\ZH{-4}{\ZK{`4}}}a:m\ZV{12}{\ZK{`3}}ua:Z\ZV{12}{\ZK{`1}}a:ne\ZV{12}{\ZK{`2}}a:va 
 b.r\ZV{12}{\ZK{`3}}ua:va:a:`Na:e\ZV{12}{\ZK{`2}}a :d\ZH{-6}{e}\ZV{12}{\ZH{-6}{\ZK{`3}}}:va:e\ZV{12}{\ZK{`2}}a :d\ZH{-6}{e}\ZV{12}{\ZH{-6}{\ZK{`3}}}:va:\ZV{12}{\ZK{`2}}a:na:M\ZV{12}{\ZK{`3}}a 
 .j\ZV{12}{\ZK{`1}}a:Y3a:n\ZV{12}{\ZK{`2}}a:ma:a ;Y1a:va:va:Y2a:k2\ZS{12}@A \\
m\ZV{12}{\ZK{`1}}a:Y1a:h\ZV{12}{\ZH{-6}{\ZK{`2}}}:v.ra:t\ZV{12}{\ZK{`3}}aH\ZS{2} Z\ZV{12}{\ZK{`1}}ua:Y3a:.c\ZV{12}{\ZK{`2}}a:ba:nDuaH\ZS{2} 
 :pa:a:v\ZV{12}{\ZK{`3}}a:k\ZV{12}{\ZH{-12}{\ZK{`2}}}\ZS{2}H\ZS{2} :p\ZV{12}{\ZK{`3}}a:d:\ZV{12}{\ZK{`1}}a v\ZV{12}{\ZK{`2}}a.=;\ZV{12}{\ZK{`3}}a:h:e\ZV{12}{\ZK{`2}}a 
 `A\ZV{12}{\ZH{-12}{\ZK{`6}}}:Bye\ZV{12}{\ZH{-4}{\ZK{`4}}}a:Y2a:t\ZV{12}{\ZK{`3}}a \ZS{2}:=\ZH{-6}{e}\ZV{12}{\ZH{-6}{\ZK{`1}}};B\ZV{12}{\ZK{`2}}a:n,a\ZS{12}@A\ZS{6}@A 524\ZS{12}@A\ZS{6}@A

;Y2a:t\ZV{12}{\ZK{`3}}a:s1\ZM{aLeDDr}:a:e\ZV{12}{\ZK{`1}}a va:a:.c\ZV{12}{\ZH{-4}{\ZK{`4}}}a I\ZH{-6}{R};=;ya:Y2a:t\ZV{12}{\ZK{`3}}a :pr\ZV{12}{\ZK{`1}}a 
 va:Y1a:h3\ZM{nCneCe}\ZV{12}{\ZH{-8}{\ZK{`4}}}+.`r\ZH{-6}{R}\ZV{12}{\ZH{-6}{\ZK{`3}}}:t\ZV{12}{\ZK{`1}}a:s1y\ZV{12}{\ZK{`2}}a 
 ;Di5a:\ZV{12}{\ZK{`3}}a:Y2a:tM\ZV{12}{\ZK{`1}}a b.ra:h2\ZV{12}{\ZH{-10}{\ZK{`4}}}:`Na:ea ma:ni5a:\ZV{12}{\ZK{`3}}a:Sa:\ZV{12}{\ZK{`2}}a:m,a\ZS{12}@A \\
ga:\ZV{12}{\ZK{`1}}a:va:e\ZV{12}{\ZK{`2}}a ya:\ZH{0}{i0///Y7}a:nt\ZV{12}{\ZK{`3}}a ga:e\ZV{12}{\ZK{`1}}a:p\ZV{12}{\ZK{`2}}a:Y2a:tMa :p\ZV{12}{\ZK{`3}}\ZV{4}{x}a:.cC\ZV{12}{\ZH{-6}{\ZK{`2}}}+ma:\ZV{12}{\ZK{`2}}a:na:\ZV{12}{\ZK{`3}}aH\ZS{2} 
 .sa:e\ZV{12}{\ZK{`1}}a:mM\ZV{12}{\ZK{`2}}a ya:\ZH{0}{i0///Y7}a:nta m\ZV{12}{\ZK{`3}}a:t\ZV{12}{\ZK{`1}}a:ya:e\ZV{12}{\ZK{`2}}a va:a:va:Za:\ZV{12}{\ZK{`3}}a:na:\ZV{12}{\ZK{`2}}aH\ZS{2}\ZS{12}@A\ZS{6}@A 525\ZS{12}@A\ZS{6}@A

`A\ZV{12}{\ZH{-6}{\ZK{`3}}}:s1y\ZV{12}{\ZK{`2}}a :pre\ZV{12}{\ZK{`3}}a:Sa:\ZV{12}{\ZK{`2}}a :h\ZH{-6}{e}\ZV{12}{\ZH{-6}{\ZK{`3}}}:m\ZV{12}{\ZK{`1}}a:na:\ZV{12}{\ZK{`2}}a 
 :p\ZV{12}{\ZK{`3}}Ua:y\ZV{12}{\ZK{`1}}a:ma:\ZV{12}{\ZK{`2}}a:na:ea :d\ZH{-6}{e}\ZV{12}{\ZH{-6}{\ZK{`3}}}:va:e\ZV{12}{\ZK{`2}}a :d\ZH{-6}{e}\ZV{12}{\ZH{-6}{\ZK{`3}}}:ve\ZV{12}{\ZK{`2}}a:Y5a:B\ZV{12}{\ZK{`3}}aH\ZS{2} 
 .s\ZV{12}{\ZK{`1}}a:m\ZV{12}{\ZK{`2}}a:p\ZV{4}{x}a:k2\ZV{12}{\ZH{-10}{\ZK{`3}}} .=\ZV{12}{\ZH{-6}{\ZK{`1}}};s\ZV{12}{\ZK{`2}}a:m,a\ZS{12}@A \\
.s\ZV{12}{\ZK{`3}}ua:t\ZV{12}{\ZK{`2}}aH\ZS{2} :p\ZV{12}{\ZK{`3}}a:Y1a:v\ZV{12}{\ZK{`2}}a.t3M\ZV{12}{\ZK{`3}}a :p\ZV{12}{\ZK{`1}}a:yyeR\ZV{12}{\ZK{`2}}a:Y2a:t\ZV{12}{\ZK{`3}}a 
 \ZS{2}:=\ZH{-6}{e}\ZV{12}{\ZH{-6}{\ZK{`1}}};B\ZV{12}{\ZK{`2}}a:\ZH{0}{i0//////Y7}a:nm1\ZM{cLe}.\ZV{12}{\ZK{`3}}a:te\ZV{12}{\ZK{`2}}a:v\ZV{12}{\ZK{`3}}a .s\ZV{12}{\ZK{`1}}a:d2\ZM{j0mC0eH0E}+a\ZV{12}{\ZH{-6}{\ZK{`2}}} 
 :pa:Z\ZV{12}{\ZK{`3}}ua:m\ZV{12}{\ZK{`2}}a:\ZH{0}{i0///Y7}a:nt\ZV{12}{\ZK{`1}}a h:e\ZV{12}{\ZK{`2}}a:ta:a\ZS{12}@A\ZS{6}@A 526\ZS{12}@A\ZS{6}@A

.sa:e\ZV{12}{\ZK{`1}}a:m\ZV{12}{\ZK{`2}}aH\ZS{2} :pa:va:tea .ja:Y3a:n\ZV{12}{\ZK{`3}}a:ta:\ZV{12}{\ZK{`1}}a m\ZV{12}{\ZK{`2}}a:ti5a:\ZV{12}{\ZK{`3}}a:na:M\ZV{12}{\ZK{`1}}a 
 .j\ZV{12}{\ZK{`2}}a:Y3a:n\ZV{12}{\ZK{`3}}a:ta:\ZV{12}{\ZK{`2}}a ;Y1a:d\ZV{12}{\ZH{-6}{\ZK{`3}}}:va:e\ZV{12}{\ZK{`1}}a .j\ZV{12}{\ZK{`2}}a:Y3a:n\ZV{12}{\ZK{`3}}a:ta:\ZV{12}{\ZK{`1}}a 
 :p\ZV{12}{\ZK{`2}}\ZV{4}{x}a:Y3a:T\ZV{12}{\ZK{`3}}a:v.ya:\ZV{12}{\ZK{`2}}aH\ZS{2}\ZS{12}@A \\
.j\ZV{12}{\ZK{`3}}a:Y3a:na:ta:\ZV{12}{\ZK{`1}}a:gnea.jR\ZV{12}{\ZH{-4}{\ZK{`4}}}a:Y3a:n\ZV{12}{\ZK{`3}}a:ta:\ZV{12}{\ZK{`1}}a .sUa:yR\ZV{12}{\ZH{-4}{\ZK{`4}}}a:s1ya 
 .ja:Y3a:n\ZV{12}{\ZK{`3}}a:te\ZV{12}{\ZK{`1}}a:nd\ZP{-8}{-4}{@R}\ZV{12}{\ZH{-6}{\ZK{`2}}};s1ya .ja:Y3a:n\ZV{12}{\ZK{`3}}a:ta:e\ZV{12}{\ZK{`1}}a:ta ;Y1a:va:S`Na:e\ZV{12}{\ZH{-4}{\ZK{`4}}}aH\ZS{2}\ZS{12}@A\ZS{6}@A 527\ZS{12}@A\ZS{6}@A

`A\ZV{12}{\ZH{-6}{\ZK{`3}}}:Y5a:B\ZV{12}{\ZK{`1}}a ;Y1a.t3\ZV{12}{\ZK{`2}}a:p\ZV{12}{\ZK{`3}}\ZV{4}{x}a:S2\ZH{-6}{M}\ZV{12}{\ZH{-6}{\ZK{`1}}} v\ZV{4}{x}a:S\ZV{12}{\ZH{-4}{\ZK{`4}}}a:`NMa 
 .va:ya:e\ZV{12}{\ZK{`3}}a:Da:\ZV{12}{\ZK{`1}}a:m\ZV{12}{\ZK{`2}}a:\ZM{0PqPOM0bgNaENdE}*:+e\ZV{12}{\ZK{`3}}a:Y1a:S\ZV{12}{\ZK{`1}}a:`N\ZV{12}{\ZK{`2}}a:ma:va:a:va:Za:nt\ZV{12}{\ZK{`3}}a 
 va:\ZV{12}{\ZK{`1}}a:`Ni5a:\ZV{12}{\ZK{`2}}aH\ZS{2} \\
v\ZV{12}{\ZK{`2}}a:na:\ZV{12}{\ZK{`3}}a b\ZV{12}{\ZK{`1}}a:sa:\ZV{12}{\ZK{`2}}a:na:e\ZV{12}{\ZK{`3}}a v\ZV{12}{\ZK{`1}}a.r8\ZV{12}{\ZH{-10}{\ZK{`2}}}+.`Na:e\ZV{12}{\ZK{`3}}a 
 n\ZV{12}{\ZH{-4}{\ZK{`5}}}a ;Y5a:sa:nD\ZV{12}{\ZK{`3}}ua:Y1a:vR\ZV{12}{\ZK{`1}}a .=\ZV{12}{\ZH{-6}{\ZK{`2}}};t2\ZV{12}{\ZK{`3}}a:Da:\ZV{12}{\ZK{`1}}a
 d\ZV{12}{\ZH{-6}{\ZK{`2}}}:ya:te\ZV{12}{\ZK{`3}}a va:\ZV{12}{\ZK{`1}}a:ya:R\ZV{12}{\ZK{`2}}a:Y4a:`Na\ZS{12}@A\ZS{6}@A 528\ZS{12}@A\ZS{6}@A

}
\vfill

\addtolength{\parskip}{-4mm}
\renewcommand{\baselinestretch}{1}
\end{minipage}

{\skt }






\clearpage

%%%%%%%%%%%%%%%%%%%%%%%%%%%%%%%%%%%%%%%%%%%%%%%%%%%%%%%%%%%%%%%%%%%%%%%%%

\clearpage

\setlength{\unitlength}{1mm}
\begin{picture}(136,192)
\put(0,130.5){\begin{minipage}{50mm}
\renewcommand{\arraystretch}{1.2}
\addtolength{\tabcolsep}{-.7mm}
\begin{tabular}{|c|c|c|}
\hline
{\small input} & {\small skt} & {\small skti} \\
\hline
{\tt a}    & {\skt A} or implicit                    & {\skti a}          \\
{\tt aa}   & {\skt A.a} or {{}\skt ;a}                & {\skti \=a}         \\
{\tt i}    & {\skt I} or {{}\skt ;i8a}               & {\skti i}          \\
{\tt ii}   & {\skt I\ZH{-6}{R}} or \hspace*{1mm} {{}\skt i8\ZS{10};a} & {\skti \={\i}} \\
{\tt u}    & {\skt o} or \hspace*{1mm} {{}\skt u}    & {\skti u}          \\
{\tt uu}   & {\skt `o} or \hspace*{1mm} {{}\skt U}   & {\skti \=u}         \\
{\tt .r}   & {\skt `x} or \hspace*{1mm} {{}\skt x}   & {\skti r\llap{\d{\kern.51em}}}         \\
{\tt .r.r} & {\skt `X} or \hspace*{1mm} {{}\skt X} & {\skti \=r\llap{\d{\kern.51em}}}       \\
{\tt .l}   & {\skt `w} or \hspace*{1mm} {{}\skt w}   & {\skti \d l}         \\
{\tt .l.l} & {\skt `W} or \hspace*{1mm} {{}\skt W} & {\skti \d{\=l}}       \\
{\tt e}    & {\skt O;} or \hspace*{1mm} {{}\skt e}    & {\skti e}          \\
{\tt ai}   & {\skt Oe;} or \hspace*{1mm} {{}\skt E}   & {\skti ai}         \\
{\tt o}    & {\skt A.ea} or \hspace*{1mm} {{}\skt *ea}  & {\skti o}          \\
{\tt au}   & {\skt A.Ea} or \hspace*{1mm} {{}\skt *Ea} & {\skti au}         \\
{\tt a.m}  & {\skt A\ZH{-6}{M}}                              & {\skti a\d m}        \\
{\tt a"m}  & {\skt A\ZH{-6}{M}}                              & {\skti a\.m}        \\
{\tt a\raisebox{-.35ex}{\tt \~\null}m}
           & {\skt A>}                              & {\skti a\d{\~m}}        \\
{\tt a.h}  & {\skt AH\ZS{2}}                              & {\skti a\d h}        \\
\hline
\end{tabular}
\end{minipage}
}
\put(57,102){\begin{minipage}{35mm}
\renewcommand{\arraystretch}{1.03}
\begin{tabular}{|c|c|c|}
\hline
{\small input} & {\small skt} & {\small skti} \\
\hline
{\tt ka}      & {\skt k}   & {\skti ka}   \\
{\tt kha}     & {\skt Ka}  & {\skti kha}  \\
{\tt ga}      & {\skt ga}   & {\skti ga}   \\
{\tt gha}     & {\skt ;Ga}  & {\skti gha}  \\
{\tt "na}     & {\skt .z}  & {\skti \.na}  \\
{\tt ca}      & {\skt ..ca}   & {\skti ca}   \\
{\tt cha}     & {\skt C}  & {\skti cha}  \\
{\tt ja}      & {\skt .ja}   & {\skti ja}   \\
{\tt jha}     & {\skt Ja}  & {\skti jha}  \\
{\tt \raisebox{-.35ex}{\tt \~\null}na}
              & {\skt Va}  & {\skti \~na}  \\
{\tt .ta}     & {\skt f}  & {\skti \d ta}  \\
{\tt .tha}    & {\skt F} & {\skti \d tha} \\
{\tt .da}     & {\skt .q}  & {\skti \d da}  \\
{\tt .dha}    & {\skt Q} & {\skti \d dha} \\
{\tt .na}     & {\skt :Na}  & {\skti \d na}  \\
{\tt ta}      & {\skt ta}   & {\skti ta}   \\
{\tt tha}     & {\skt Ta}  & {\skti tha}  \\
{\tt da}      & {\skt d}   & {\skti da}   \\
{\tt dha}     & {\skt ;Da}  & {\skti dha}  \\
{\tt na}      & {\skt na}   & {\skti na}   \\
{\tt pa}      & {\skt :pa}   & {\skti pa}   \\
{\tt pha}     & {\skt :P}  & {\skti pha}  \\
{\tt ba}      & {\skt ba}   & {\skti ba}   \\
{\tt bha}     & {\skt Ba}  & {\skti bha}  \\
{\tt ma}      & {\skt ma}   & {\skti ma}   \\
{\tt ya}   & {\skt ya}   & {\skti ya}   \\
{\tt ra}   & {\skt .=}   & {\skti ra}   \\
{\tt la}   & {\skt l}   & {\skti la}   \\
{\tt va}   & {\skt va}   & {\skti va}   \\
{\tt "sa}  & {\skt Za}  & {\skti \'sa}  \\
{\tt .sa}  & {\skt :Sa}  & {\skti \d sa}  \\
{\tt sa}   & {\skt .sa}   & {\skti sa}   \\
{\tt ha}   & {\skt h}   & {\skti ha}   \\
\hline
\end{tabular}
\end{minipage}}
\put(7,39){\begin{minipage}{35mm}
\renewcommand{\arraystretch}{1.03}
\begin{tabular}{|c|c|c|}
\hline
{\small input} & {\small skt} & {\small skti} \\
\hline
{\tt 0}    & {\skt 0}    & {\skti 0}    \\
{\tt 1}    & {\skt 1}    & {\skti 1}    \\
{\tt 2}    & {\skt 2}    & {\skti 2}    \\
{\tt 3}    & {\skt 3}    & {\skti 3}    \\
{\tt 4}    & {\skt 4}    & {\skti 4}    \\
{\tt 5}    & {\skt 5}    & {\skti 5}    \\
{\tt 6}    & {\skt 6}    & {\skti 6}    \\
{\tt 7}    & {\skt 7}    & {\skti 7}    \\
{\tt 8}    & {\skt 8}    & {\skti 8}    \\
{\tt 9}    & {\skt 9}    & {\skti 9}    \\
\hline
\end{tabular}
\end{minipage}
}
\put(99,89.2){\begin{minipage}{38mm}
\renewcommand{\arraystretch}{.95} %{1.16}
{\skt }
\begin{tabular}{|c|c|c|}
\hline
{\small input} & {\small skt} & {\small skti} \\
\hline
{\tt .a}      & {\relsize{-1}{{}\skt\symbol{89}}}
                                             & {{}\skti $^\prime $}  \\
{\tt |}       & {\relsize{-1}{\skt \ZS{12}@A}}       & {\skti {\upshape\,$\mid$}}    \\
{\tt ||}      & {\relsize{-1}{\skt \ZS{12}@A\ZS{6}@A}}      & {\skti {\upshape\,$\mid\mid$}}   \\
{\tt .o}      & {\relsize{-1}{\skt ?}}      & {\skti O\~m}   \\
{\tt "h}      & {\relsize{-1}{\skt H1}}      & {\skti \ZZ h}   \\
{\tt "da}     & {\relsize{-1}{\skt L}}     & {\skti \ZZ da}  \\
{\tt "dha}    & {\relsize{-1}{\skt L1h}}    & {\skti \ZZ dha} \\
{\tt a@}      & {\relsize{-1}{\skt A\ZM{FTV}\ZS{20}}}      & {\skti a{\upshape$^\circ$}}   \\
{\tt ..}      & {\relsize{-1}{\skt \ZN{.}}}      & {\skti .}   \\
{\tt a\#}     & {\relsize{-1}{\skt A\ZH{-6}{<}}}      & {\skti \ZA{30}{a}}   \\
{\tt a!}      & {\relsize{-1}{\skt A\ZH{-6}{\ZK{`7}}}}      & {\skti \ZA{1}{a}}   \\
{\tt a!!}     & {\relsize{-1}{\skt A\ZH{-9}{\ZK{`7}}\ZH{-3}{\ZK{`7}}}}     & {\skti \ZA{6}{a}}  \\
{\tt a\_}     & {\relsize{-1}{\skt A\ZH{-6}{\ZK{`8}}}}      & {\skti \ZA{2}{a}}   \\
{\tt a"1}     & {\relsize{-1}{\skt A\ZK{1\ZH{-12}{`7}\ZH{-10}{`8}}}}     & {\skti \ZA{4}{a}}\rule{0mm}{6mm} \\
{\tt a"3}     & {\relsize{-1}{\skt A\ZH{-6}{\ZK{`8}}\ZK{3\ZH{-12}{`7}\ZH{-8}{`8}}}}     & {\skti \ZA{3}{a}}\rule{0mm}{6mm} \\
{\tt a.1}     & {\relsize{-1}{\skt A\ZK{1\ZH{-10}{`8}}}}     & {\skti \ZA{15}{a}}\rule{0mm}{5mm} \\
{\tt a.3}     & {\relsize{-1}{\skt A\ZK{3\ZH{-12}{`7}\ZH{-8}{`8}}}}     & {\skti \ZA{16}{a}}\rule{0mm}{6mm} \\
{\tt a\^{}}   & {\relsize{-1}{\skt A\ZH{-6}{\ZK{`0}}}}      & {\skti \ZA{5}{a}}\rule{0mm}{6mm} \\
{\tt a'}      &  ---                         & {\skti \ZA{8}{a}}      \\
{\tt a`}      &  ---                         & {\skti \ZA{7}{a}}      \\
{\tt a<1>}    & {\relsize{-1}{\skt A\ZH{-6}{\ZK{`1}}}}    & {\skti \ZA{9}{a}}    \\
{\tt a<2>}    & {\relsize{-1}{\skt A\ZH{-6}{\ZK{`2}}}}    & {\skti \ZA{10}{a}}    \\
{\tt a<3>}    & {\relsize{-1}{\skt A\ZH{-6}{\ZK{`3}}}}    & {\skti \ZA{11}{a}}    \\
{\tt a<2r>}   & {\relsize{-1}{\skt A\ZH{-10}{\ZK{`4}}}}   & {\skti \ZA{12}{a}}   \\
{\tt a<2u>}   & {\relsize{-1}{\skt A\ZH{-10}{\ZK{`5}}}}   & {\skti \ZA{13}{a}}   \\
{\tt a<3k>}   & {\relsize{-1}{\skt A\ZH{-12}{\ZK{`6}}}}   & {\skti \ZA{14}{a}}   \\
{\tt a<!!>}   & \raisebox{0mm}[\height][0mm]{\relsize{-1}{\skt A\ZH{-11}{\ZK{`!}}}}   & {\skti \ZA{25}{a}}   \\
{\tt a<u>}    & \raisebox{0mm}[\height][0mm]{\relsize{-1}{\skt A\ZH{-6}{\ZK{`u}}}}    & {\skti \ZA{17}{a}}    \\
{\tt a<w>}    & \raisebox{0mm}[\height][0mm]{\relsize{-1}{\skt A\ZH{-6}{\ZK{`z}}}}    & {\skti \ZA{18}{a}}    \\
{\tt a<\_>}   & \raisebox{0mm}[\height][0mm]{\relsize{-1}{\skt A\ZP{-16}{-10}{\ZK{@I@o}}}}    & {\skti \ZA{19}{a}}    \\
{\tt a<.>}    & \raisebox{0mm}[\height][0mm]{\relsize{-1}{\skt A\ZP{-8}{-10}{\ZK{@M}}}}    & {\skti \ZA{20}{a}}    \\
{\tt a<..>}   & \raisebox{0mm}[\height][0mm]{\relsize{-1}{\skt A\ZP{-8}{-10}{\ZK{@M\ZS{-9}@M}}}}   & {\skti \ZA{21}{a}}   \\
{\tt a<!>}    & \raisebox{0mm}[\height][0mm]{\relsize{-1}{\skt A\ZP{-6}{-12}{\ZK{@I\ZV{2}{@I}}}}}    & {\skti \ZA{22}{a}}    \\
{\tt a<\^{}>} & \raisebox{0mm}[\height][0mm]{\relsize{-1}{\skt A\ZP{-6}{-16}{\ZK{@r\ZP{-3}{5}{@b}\ZV{2}{@b}}}}}    & {\skti \ZA{24}{a}}    \\
{\tt a<s>}    & \raisebox{0mm}[\height][0mm]{\relsize{-1}{\skt A\ZK{`s}}}    & {\skti \ZA{23}{a}}    \\
\hline
\end{tabular}
{\skt }
\end{minipage}
}
\put(45,3){\makebox(0,0){\textbf{\large Encoding and Transliteration Scheme.}}}
\put(42,-3){\makebox(0,0){(default options)}}
\end{picture}

\clearpage






\section{Vedic Accent Marking}%
%
Since the significance of the {\skti devan\=agar\={\i}} accent marking may
differ between {\skti Veda\/}s (e.g.~{\skt A\ZH{-6}{\ZK{`7}}} is {\skti svarita} in
{\skti \d Rgveda}, but {\skti ud\=atta} in {\skti K\=a\d thaka\/}), the character
used for encoding is, where possible, vaguely similar to the form used in
the {\skti devan\=agar\={\i}}, and this form is largely retained in the 
transliterated output. 
\vspace{\fill}

{\skt }
\renewcommand{\arraystretch}{1}%{1.2}
\begin{tabular}{|c|c|c|c|}
\multicolumn{4}{c}{{\skti \d Rgveda\/} etc.~marking system:} \\[.5mm]
\multicolumn{4}{c}{use option~6 to enable these in}\\[-.5mm]
\multicolumn{4}{c}{the basic transliteration mode.} \\[2mm]
\hline input & skt     & skti          & sktt       \\
\hline
\verb+a!+    & {\skt A\ZH{-6}{\ZK{`7}}}    & {\skti \ZA{1}{a}}    & error      \rule{0mm}{6.5mm} \\
\verb+a!!+   & {\skt A\ZH{-9}{\ZK{`7}}\ZH{-3}{\ZK{`7}}}   & {\skti \ZA{6}{a}}   & error      \rule{0mm}{6.5mm} \\
\verb+a_+    & {\skt A\ZH{-6}{\ZK{`8}}}    & {\skti \ZA{2}{a}}    & {\small see note} \rule{0mm}{5.5mm}\\
\verb+a"1+   & {\skt A\ZK{1\ZH{-12}{`7}\ZH{-10}{`8}}}   & {\skti \ZA{4}{a}}   & error      \rule{0mm}{7.5mm}\\
\verb+a"3+   & {\skt A\ZH{-6}{\ZK{`8}}\ZK{3\ZH{-12}{`7}\ZH{-8}{`8}}}   & {\skti \ZA{3}{a}}   & error      \rule{0mm}{7.5mm}\\
\verb+a.1+   & {\skt A\ZK{1\ZH{-10}{`8}}}   & {\skti \ZA{15}{a}}   & error      \rule{0mm}{5.5mm}\\
\verb+a.3+   & {\skt A\ZK{3\ZH{-12}{`7}\ZH{-8}{`8}}}   & {\skti \ZA{16}{a}}   & error      
             \raisebox{-2.5mm}{\rule{0mm}{10mm}}\\
\hline
\multicolumn{4}{c}{Note: {\smaller The underscore character is}
                   \rule{0mm}{6mm}}\\[-.8mm]
\multicolumn{4}{c}{{\smaller is used differently in technical mode.}}\\
\multicolumn{4}{c}{Western (and technical) marking.}\rule{0mm}{12mm} \\[2mm]
\hline input & skt     & skti          & sktt       \\
\hline
\verb+a^+    & {\skt A\ZH{-6}{\ZK{`0}}}    & {\skti \ZA{5}{a}}    & {\sktt \ZA{5}{a}} \rule{0mm}{7mm} \\
\verb+a'+    & error                     & {\skti \ZA{8}{a}}    & {\sktt \ZA{8}{a}} \\
\verb+a`+    & error                     & {\skti \ZA{7}{a}}    & {\sktt \ZA{7}{a}} \\\hline
\end{tabular}
%
\hspace{\fill}
%
\begin{tabular}{|c|c|c|c|}
\multicolumn{4}{c}{{\skti S\=amaveda} and other accents: }\\[.5mm]
\multicolumn{4}{c}{use option~7 to enable these in}\\[-.5mm]
\multicolumn{4}{c}{the basic transliteration mode.} \\[2mm]
\hline input & skt     & skti          & sktt       \\
\hline
\verb+a<1>+  & {\skt A\ZH{-6}{\ZK{`1}}}   & {\skti \ZA{9}{a}}  & error \rule{0mm}{7mm} \\
\verb+a<2>+  & {\skt A\ZH{-6}{\ZK{`2}}}   & {\skti \ZA{10}{a}}  & error \rule{0mm}{6mm} \\
\verb+a<3>+  & {\skt A\ZH{-6}{\ZK{`3}}}   & {\skti \ZA{11}{a}}  & error \rule{0mm}{6mm} \\
\verb+a<2r>+ & {\skt A\ZH{-10}{\ZK{`4}}}  & {\skti \ZA{12}{a}} & error \rule{0mm}{6mm} \\
\verb+a<2u>+ & {\skt A\ZH{-10}{\ZK{`5}}}  & {\skti \ZA{13}{a}} & error \rule{0mm}{6mm} \\
\verb+a<3k>+ & {\skt A\ZH{-12}{\ZK{`6}}}  & {\skti \ZA{14}{a}} & error \rule{0mm}{6mm} \\\hline
\verb+a<!!>+ & {\skt A\ZH{-11}{\ZK{`!}}}  & {\skti \ZA{25}{a}} & error \rule{0mm}{7mm} \\
\verb+a<u>+  & {\skt A\ZH{-6}{\ZK{`u}}}   & {\skti \ZA{17}{a}}  & error \rule{0mm}{5mm} \\
\verb+a<w>+  & {\skt A\ZH{-6}{\ZK{`z}}}   & {\skti \ZA{18}{a}}  & error \rule{0mm}{5mm} \\
\verb+a<_>+  & {\skt A\ZP{-16}{-10}{\ZK{@I@o}}}   & {\skti \ZA{19}{a}}  & error \rule{0mm}{5mm} \\
\verb+a<.>+  & {\skt A\ZP{-8}{-10}{\ZK{@M}}}   & {\skti \ZA{20}{a}}  & error \rule{0mm}{5mm} \\
\verb+a<..>+ & {\skt A\ZP{-8}{-10}{\ZK{@M\ZS{-9}@M}}}  & {\skti \ZA{21}{a}} & error \rule{0mm}{5mm} \\
\verb+a<!>+  & {\skt A\ZP{-6}{-12}{\ZK{@I\ZV{2}{@I}}}}   & {\skti \ZA{22}{a}}  & error \rule{0mm}{5mm} \\
\verb+a<^>+  & {\skt A\ZP{-6}{-16}{\ZK{@r\ZP{-3}{5}{@b}\ZV{2}{@b}}}}   & {\skti \ZA{24}{a}}  & error \rule{0mm}{5mm} \\
\verb+a<s>+  & {\skt A\ZK{`s}}   & {\skti \ZA{23}{a}}  & error \rule{0mm}{5mm} \\\hline
\end{tabular}
{\skt }
\vspace{\fill}

The pre-processor only allows accents after a vowel, Vedic 
{\skti anusv\=ara}, or~\verb+n+; where the vowel is nasalised 
as well, the nasalisation must be after the accent, 
i.e.~\verb+a_#+ to produce {\skt A\ZH{-6}{<}\ZH{-6}{\ZK{`8}}}.

The {\skti devan\=agar\={\i}\/} accent marks themselves may be printed in
colour (red is the tradition) or grey. This actually makes the text easier 
to read: the sample text from the {\skti \d Rgveda} on page~\pageref{rgtext}
has the accents in the first four verses the default black, and in the next 
four verses they are `lightgray' (using \verb+xcolor+).
To set the accent colour redefine the \verb+\ZK+ command anywhere in the 
\verb+.skt+~file as \verb+\renewcommand{\ZK}[1]{{\lightgray #1}}+ for example.

In transiteration, the accents should stack correctly and track in 
italics, e.g.:\\
{\sktt \ZA{68}{\ZW{\=r\llap{\d{\kern.51em}}}}} and {\sktu \ZA{68}{\ZW{\=r\llap{\d{\kern.51em}}}}} (signifying a long or short 
{\skti ud\=atta r\llap{\d{\kern.51em}}} used for pronunciation only).


\end{document}

